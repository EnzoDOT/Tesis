\pagestyle{fancy}
\chapter{Conclusiones}
\label{cap:conc}
\lhead{\thepage}
\rhead{Conclusiones} 
\vspace{0.01\textheight}
\pagebreak

\clearpage \pagebreak En este trabajo hemos desarrollado métodos numéricos eficientes 
para la resolución de los problemas directos e inversos con 
aplicaciones en tomografía óptica. 
Si bien el foco de esta tesis ha sido el desarrollo de algorítmos 
para la resolución de la ETR de forma eficiente en el campo 
de tomografía óptica, 
dado que la ETR es una ecuación de Boltzmann 
linearizada de utilidad en el modelado de diversos 
fenómenos físicos, estos algorítmos pueden ser aplicados e impactan 
en otras áreas de la ciencia y la tecnología de gran interés~\cite{Howell2010, Thynell1998,Duderstadt1979,Qin2015,Dymond1997,Chandrasekhar1960,Zhu2005,Zhu2010,Vassiliev2010,Bedford2019,Vassiliev2010,Bedford2019,Larsen2006, Sanchez1982, Anli2006,Mishchenko1999, Prasher2003}

 Hemos desarrollado algoritmos que permiten obtener soluciones 
 a la ETR de forma eficiente, en entornos de máquinas paralelas, 
 y que para el caso de soluciones suaves presentan un alto órden de convergencia. 
 Demostramos que los códigos desarrollados en esta tésis son matemáticamente 
 correctos mediante comparación con soluciones manufacturadas y soluciones 
 analíticas, así como también demostramos que son físicamente 
 precisos mediante comparación con resultados experimentales 
 reportados en la literatura.
 
 En aplicaciones realistas, surge el fenómeno de capas límite exponenciales 
 estudiado en la sección~\ref{sec:blayer}, el cual representa un desafío 
 vigente, que a consideración del autor no ha sido correctamente 
 abordado en la literatura. Por esta razón, en esta tesis 
 se desarrollo la teoría de capa límite correspondiente, se demostró 
 el caracter exponencial de esta capa límite, el cual se manifiesta 
 incluso para problemas altamente difusivos.
 
 Hemos demostrado que es posible 
 tener alto órden de convergencia en situaciones realistas para 
 el caso dependiente del tiempo en una única dimensión espacial, 
 explotando el análisis asintótico desarrollado en esta tésis para las capas límite. 
 Como resultado de esta Tesis, 
 surge como tema de investigación a futuro, la generalización y el desarrollo de algoritmos capaces 
 de resolver las estructuras de capa límite para el caso 
 multidimensional de la ETR. Si bien este último problema no fue abordado, 
 la teoría desarrollada en el marco de esta Tesis sienta las bases 
 para el desarrollo de algoritmos de alto órden de convergencia en el 
 caso multidimensional, de interés e impácto en diversas 
 áreas de la ciencia y la tecnología. 
 
 Hemos extendido el uso del algorítmo FC--DOM desarrollado 
 para el problema directo, al problema inverso en tomografía óptica. 
 Se presentó una formulación mediánte el método adjunto para 
 la obtención del gradiente de la función objetivo, que incluye 
 las condiciones de borde de Fresnel. 
 Resolvimos el problema inverso en tomografía óptica mediante el uso 
 del método cuasi-Newton lm-BFGS.
 Demostramos que es posible obtener una aceleración adicional 
 en las reconstrucciones del parámetro de absorción 
 mediante la estrategia de Fuentes Múltiples Superpuestas. 
En el método FMS, las múltiples fuentes láser presentan una activación de 
 forma sincronizada, que permite optimizar y ganar control en los tiempos computacionales requeridos para las simulaciones. 
 Mostramos que esta estrategia produce una aceleración en un factor de 
 seis y siete para los casos estudiados, produciendo imágenes 
 de similar calidad a las obtenidas mediánte el Método de Barrido 
 utilizado ubicuamente en la literatura. 

 En el marco del problema inverso, posibles direcciones futuras 
 de investigación incluyen el estudio del impacto 
 de la estructura de capa límite en las reconstrucciones, 
 y la generalización de los algoritmos presentados en esta Tesis 
 a otros problemas relacionados, como lo es la obtención 
 simultánea de los coeficientes de absorción y dispersión, 
 o la aplicación a problemas de tomografía por fluorescencia~\cite{Klose2009,Ren2010}.


\pagestyle{empty}
