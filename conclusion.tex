\pagestyle{fancy}
\chapter{Conclusiones}
\label{cap:conc}
\lhead{\thepage}
\rhead{Conclusiones} 
\vspace{0.01\textheight}


\clearpage \pagebreak En este trabajo se desarrollaron estrategias computacionales para el 
estudio de problemas de transporte de partículas neutras, y la 
caracterización de los medios que estas atraviesan.
En particular, hemos estudiado radiación de fotones y 
flujos de neutrones.

El objetivo principal ha sido la resolución del 
problema inverso, que consiste en determinar las características 
del medio participante mediante los datos obtenidos por mediciones 
de radiación producida por fuentes externas.
Nos hemos enfocado en diversos ejemplos de 
reconstrucción de las características físicas de tejidos humanos, 
incluyendo tumores y activaciones hemodinámicas.
La resolución de este tipo de problemas exige una enorme cantidad 
de recursos computacionales, ya que requiere la propagación temporal 
de la radiación en el medio durante muchas iteraciones, 
hasta minimizar las diferencias entre las predicciones teóricas y 
los valores medidos en los detectores. 
Esta altísima demanda de cálculos limita drásticamente 
la aplicación de los métodos precisos, imponiendo la necesidad de 
recurrir a aproximaciones que no siempre producen resultados correctos.

Hemos abordado este problema combinando tres estrategias principales, 
desarrolladas a lo largo de nuestro trabajo:

\begin{enumerate}
\item El uso de un método espectral novedoso de alto orden
para el tratamiento del problema directo, que permite resolver la 
ecuación de transporte con gran precisión y con reducido esfuerzo 
computacional. La novedad del método se basa en la incorporación 
del procedimiento de continuación de Fourier 
en ordenadas discretas (FC--DOM)~\cite{Gaggioli2019}. 
Este método permite convertir cualquier función arbitraria en periódica, y 
por ende, resolver los operadores diferenciales utilizando transformadas 
de Fourier. Con ello se logran resultados prácticamente libres de 
errores de dispersión.

\item Una efectiva estrategia de descomposición del dominio espacial 
\cite{Gaggioli2022}, 
que permite la implementación del método FC--DOM en máquinas paralelas. 
El algoritmo presentado en este trabajo muestra que utilizando 
256 procesadores, es posible lograr una eficiencia de escala 
mejor a la ideal (137\%). 
La escalabilidad lograda es mucho mayor que la de todos los 
resultados reportados hasta el momento, teniendo en cuenta que 
se aplica a problemas generales, sin restricciones a las 
características del medio.

\item Una configuración de fuentes múltiples superpuestas (FMS)~\cite{Gaggioli2022}, 
que permite resolver el problema inverso a un costo computacional 
independiente del número de fuentes empleados. 
Con esta metodología logramos acelerar significativamente la reconstrucción 
de los parámetros ópticos, que en nuestros ejemplos han dado un factor mayor a 6 
respecto a los métodos de secuencias de fuentes únicas (MB).

\end{enumerate}

A lo largo de esta Tesis, se describen diversos aspectos y resultados 
de nuestra investigación. 
Comenzamos describiendo el método FC--DOM y su implementación en la 
solución de la ecuación de transporte radiativo, con amplio detalle.
Validamos nuestra estrategia presentando diversos ejemplos, 
comparando la precisión y eficiencia del mismo con otros métodos 
recientes propuestos en la literatura.
Para ello, hemos realizado un análisis de los errores y los 
tiempos computacionales empleados en distintos escenarios.
En primer lugar, mostramos que nuestro método presenta un cuarto 
orden de aproximación en la derivadas espaciales, mientras que 
los métodos tradicionales de diferencias finitas son de tercer orden.
En segundo término, se empleó una solución manufacturada para 
validar nuestro método.
Esto significa que  
se propone una función analítica que representa a la intensidad 
de radiación, y con ella se calcula cuál sería la ``fuente manufacturada'' correspondiente.
Luego, se parte de esta fuente, y se resuelve la ecuación de transporte, 
obteniendo una intensidad lumínica que se compara con la solución 
manufacturada inicial. 
Hicimos esto simulando dos tipos de medios con propiedades ópticas similares 
al tejido humano. Estos medios tienen representación experimental, 
mediante fantomas compuestos de epoxy y tinta. 
En uno de estos arreglos experimentales, se trata de un cubo homogéneo, 
y en el otro, se introduce un anillo cilíndrico con agua en su interior.
Se lograron acuerdos excelentes entre nuestras predicciones y las 
mediciones experimentales de los detectores. El segundo arreglo mencionado 
tiene una importancia fundamental, dado que al introducir una inhomogeneidad 
no dispersiva, se excluye la posibilidad de resolución teórica mediante 
el uso de la aproximación de difusión.
Luego se analizó un problema analítico de referencia, que tiene la 
particularidad de presentar fenómeno de rayos. Este fenómeno está ligado 
exclusivamente a las variables angulares, por lo tanto, podemos eludirlo 
utilizando un grillado más fino en esta discretización. 
Finalmente, hemos validado el método a través de los resultados 
obtenidos en un problema de fuente puntual en un medio infinito, 
isótropo y homogéneo, 
para el que se conoce solución analítica.
Comparamos nuestras soluciones con las que se obtienen mediante un 
método de diferencias finitas de tercer orden, obteniendo errores 
significativamente menores.

Durante nuestra investigación nos topamos con un descubrimiento importante: 
la existencia de una estructura de capa límite en las soluciones a la 
ecuación de transporte. Esta se manifiesta en variaciones abruptas 
de las soluciones de la ETR, que aparecen en regiones espaciales 
muy cercanas a los bordes. 
Numéricamente, la existencia de esta capa límite implica una 
degradación considerable en el orden de convergencia, y otras dificultades 
como ser la aparición de oscilaciones no físicas en las soluciones. 
A lo largo de décadas, se han propuesto distintas interpretaciones a 
estos fenómenos, que han resultado erróneas.
En nuestro trabajo, realizamos las simulaciones en un contexto de flujo 
de neutrones. 
No sólo describimos las condiciones que producen la aparición de 
este fenómeno, sino que también propusimos y describimos un método 
para sobreponerse a las dificultades numéricas que este produce. 
La metodología se basa en un cambio de 
variables, que produce una discretización con mucho mayor densidad de 
puntos en los bordes. Esta discretización, combinada con el método FC--DOM, 
permite obtener resultados de muy alta precisión, con un significativo 
ahorro de los recursos computacionales requeridos (por ejemplo, 
obtenemos mucho mejores resultados en una simulación FC--DOM con 400 puntos de 
este grillado espacial, comparados a un cálculo que emplea la diferenciación de 
diamante, utiliza 10000 puntos equiespaciados, y que además presenta 
soluciones plagadas de oscilaciones espurias).

Dedicamos una sección para explicar la 
estrategia de paralelización propuesta, demostrando sus ventajas.
Si bien los problemas inversos se resuelven mayormente basandose 
en arquitecturas de GPU~\cite{Doulgerakis2017}, estos métodos 
utilizan la aproximación de difusión, que sólo puede asumirse 
válida en escenarios particulares. 
Según la referencia~\cite{Coelho2014}, que hace una revisión de los 
principales algoritmos presentados en la literatura, 
todas las estrategias de paralelización de la ETR resultan en una 
eficiencia significativamente por debajo de la ideal (al menos que se 
trate de situaciones muy particulares, restringidas a medios no 
absorbentes ni dispersantes).
En los ejemplos que presentamos, obtuvimos una escalabilidad de 
137\% empleando 256 procesadores. Logramos resolver, por ejemplo, 
un problema modelo que requiere 2 días de cálculo, en menos de 30 minutos, 
utilizando 64 procesadores.
Una reducción adicional significativa se obtuvo mediante el uso 
del método de múltiples fuentes, desarrollado en nuestra investigación.
Con el mismo, podemos lograr factores de aceleración cercanos a la 
decena, empleando un número alto de fuentes sin necesidad de 
repetir la solución del problema para cada una de ellas en forma 
independiente (tal como lo hace el ``método de barrido'' 
mayormente utilizado en tomografía óptica).

Durante nuestro trabajo hemos debido dedicar grandes esfuerzos 
al desarrollo matemático de varias expresiones, a estudios de 
convergencia, y a la elaboración de métodos matemáticos que 
permiten facilitar 
los cálculos. Específicamente, desarrollamos un método adjunto para 
el cálculo de gradientes funcionales en el contexto de tomografía óptica, 
el cual incorpora las condiciones de borde de Fresnel. 
Dado que esto merece una explicación detallada,  antes de presentar los resultados finales de problemas 
inversos, dedicamos una sección a estos desarrollos.

Finalmente, empleamos todos los algoritmos desarrollados en esta Tesis, 
para resolver el problema inverso para dos casos de prueba relacionados. 
En el primer caso, simulamos la obtención de imagenes para un tumor 
en un cullo humano. Tomamos una imagen de resonancia magnética de un 
cuello humano, y la reconstruímos fijando ciertos coeficientes de 
absorción y dispersión similares a los que se encuentran en los trabajos 
de referencia. A esa configuración le agregamos una inclusión 
(tumor) en una región del tejido blando. 
Asumimos un número determinado de láseres, que iluminan la región en 
distintos tiempos, y sabiendo las propiedades del medio, resolvimos 
el problema directo, adquiriendo la información medida por una serie 
de detectores colocados alrededor del medio en cuestión.
Esta información es utilizada entonces para el problema inverso, 
cuya misión es reconstruir las propiedades del medio, y por supuesto, 
determinar las dimensiones y posición de la inclusión. 
Los resultados obtenidos son excelentes. Logramos reconstruir con 
sumo detalle las imágenes del medio. Logramos también realizar 
estos cálculos en forma eficiente, diseñando una estrategia de 
colocación y encendido de las fuentes, que permite superar ampliamente al 
método de barrido.
El segundo caso consiste en la simulación de una cabeza humana, 
la cual incorpora la región ubicada entre el cráneo y el 
cerebro, en la que se encuentra el fluido cerebroespinal. 
Dadas las características de los coeficientes de absorción y dispersión, 
la aproximación difusiva no es válida en esta región.
Aquí también introducimos ciertas inclusiones, que pueden corresponder 
a regiones en los cuales se produce la activación hemodinámica. 
Hemos encontrado que dada la geometría del sistema, y probablemente a 
la configuración de las inclusiones, es preferible alterar el 
orden de encendido de las fuentes, por lo que concluímos que la 
ubicación y temporalidad de las mismas merece un estudio específico.
De todos modos, con el orden encontrado, logramos reproducir el 
medio con amplio detalle y precisión, en un tiempo 6 veces menor 
al empleado con el método de barrido.

Si bien en esta Tesis el desarrollo de algoritmos 
para la resolución de la ecuación de transporte en forma eficiente 
ha tenido un enfoque principal en el campo de tomografía óptica, 
nuestros algoritmos pueden ser aplicados, con alto impacto, 
en otras áreas de la ciencia y la tecnología de gran interés~\cite{Howell2010, Thynell1998,Duderstadt1979,Qin2015,Dymond1997,Chandrasekhar1960,Zhu2005,Zhu2010,Vassiliev2010,Bedford2019,Vassiliev2010,Bedford2019,Larsen2006, Sanchez1982, Anli2006,Mishchenko1999, Prasher2003}

 
\pagestyle{empty}
