\pagestyle{empty}
\chapter*{}
\addcontentsline{toc}{chapter}{Agradecimientos}

\begin{center}
\begin{large}
\textbf{Agradecimientos}
\end{large}
\end{center}

\vspace{1cm}
En primer lugar quiero y debo agradecer al Dr. Oscar Bruno, cuya 
colaboración y dedicación hicieron posible que 
este trabajo lograra la forma que hoy tiene. Oscar dedicó largas 
horas de su tiempo a trabajar conmigo, 
y este trabajo es en gran parte el resultado de esa colaboración. 
La relación profesional y de amistad a la distancia que mantuvimos 
con Oscar en todos estos años es para mí invaluable. 
Quiero agradecer también a mi director, el Dr. Darío Mitnik, por 
haber confiado en mí para hacer este trabajo, y por haberme permitido 
trabajar con una gran libertad e independencia en todos estos años. 
Agradezco a mi familia, por el apoyo constante y permanente 
en todos estos años de estudio. A la familia Osimi, en especial a 
Claudio y Marcelo. A mís tixs, primxs, sobrinxs y hermanxs.
A los Doctores Marcelo Ambrosio e Ilán Gomez, 
ambos fueron muy importantes en el inicio de mi formación de posgrado. 
Al Dr. Martin Maas, quien me facilitó el contacto con el Dr. Bruno, 
así como a la FCEyN y al Departamento de Física, y todos los docentes, 
investigadores y personas encargadas de gestionar el programa de cursos de profesores visitantes. 
Al Dr. Edwin 
Jimenez, quien me presto asistencia para que pudiera correr mis códigos 
en el clúster EMSCAT (donde se realizaron varias de las simulaciones 
presentadas en esta tesis), y al personal del IAFE, 
donde pude probar por primera vez mis códigos en un clúster.
 A les amigues y colegas del IAFE, en especial a 
Claudia Montanari, Ana Pichel, Silvina Cichowolski, Laura Suad, Maxi Sendra, Maria Silvia Gravielle, Sebastián López, 
Federico Nuevo, Sofia Burne, Silvina Cardenas, Gabriela Boscoboinik, Diego Arbó, Esteban Reisin, Rafael Ferraro,  y Ernesto Eiroa, con quienes 
compartí muchas horas de camaradería, movilizaciones, almuerzos, y hasta oficina. 
Mi paso por el IAFE no hubiera sido igual sin ustedes. 
A les colegas becaries, por la lucha sostenida durante años en reclamo de 
reconocimiento pleno a nuestro trabajo como tal. 
A CONICET, y a cada trabajador Argentino que paga los impuestos, 
sin el financiamiento de CONICET, no hubiera sido posible este trabajo. 
A todos los que me acompañaron en este largo recorrido, 
en especial a mis amigos de Coghlan, Mauri, Marito, Colo, a mis amigos y 
ex compañeros de Bahía, Kito, Paty, Juan. A mi amigo Franco Cortesi. A les que estuvieron y a les que están. A Belén, 
por acompañarme y mostrarme su mundo. A la Universidad Nacional del Sur y sus 
docentes, donde realicé mi formación de grado.

También debo agradecer a los docentes de la Facultad de Ciencias Exactas y Naturales. 
Disfruté muchísimo cada curso que hice, y valoro enormemente el trabajo 
que se hace manteniendo una educación de excelencia en las universidades públicas, 
y gratuitas de nuestro país.
Agradezco al secretario académico, 
Mariano Mayochi por su excelente trabajo. 
Y a cada trabajador de la administración pública, y de personal de apoyo de 
CONICET. A Alexandra Elbakyan, y a Sci-Hub, por garantizar el acceso 
a artículos científicos que de otra forma sería muy difícil conseguir, 
dificultando la labor científica en países en vías de desarrollo.

La ciencia es la suma del conocimiento adquirido por la humanidad 
a lo largo de generaciones. Como tal, es el resultado de un trabajo colectivo, 
que se realiza en sociedad, tanto por el trabajo mancomunado 
que atraviesa sociedades y generaciones de científicos 
que avanzan el conocimiento haciendo uso 
de esa preciada herramienta llamada ``método científico'', 
así como del resto de los trabajadores que hacen al contexto sociocultural 
que permite que ese trabajo se 
desarrolle. Esta tesis no hubiera sido posible sin ustedes, 
por eso, como diría un gran amigo al referirse 
a las ondas planas, ``desde siempre y para siempre'', gracias.
