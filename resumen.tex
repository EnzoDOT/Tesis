\pagestyle{empty}
\chapter*{}
\addcontentsline{toc}{chapter}{Resumen}

\begin{center}
\begin{large}
\textbf{Algoritmos eficientes para aplicaciones en tomografía óptica}
\end{large}
\end{center}

\vspace{1cm}
En este trabajo desarrollamos    
algoritmos eficientes para la resolución de los problemas directos 
e inversos necesarios en la disciplina de tomografía óptica. Para ello, utilizamos la Ecuación de Transporte 
Radiativo (ETR) como modelo físico de transporte para la radiación 
en la materia. 

La ETR es resuelta por medio de algoritmos basados en el Método  
de Continuación de Fourier en Ordenadas Discretas (FC-DOM, de su sigla en inglés). Estos algoritmos 
permiten resolver la ERT de forma eficiente y en entornos de máquinas paralelas 
con escalabilidad ideal, como se muestra en esta tesis. 

Adicionalmente, presentamos una identificación y caracterización de estructuras 
de capa límite existentes en las soluciones de la ETR, en conjunto con un método para su resolución. La teoría 
de capa límite presentada provee estrategias para la solución de la ETR 
que dan como resultado un 
alto orden de convergencia en todas las variables involucradas,
como se demuestra en la sección~\ref{sec:blayer} para el caso de una única dimensión espacial.

Para la resolución del problema inverso en tomografía óptica utilizamos el método de minimización 
de Broyden–Fletcher–Goldfarb–Shanno (BFGS) con uso de memoria reducido 
(lm-BFGS, de su sigla en inglés). 
Se desarrolla una metodología que mediante la resolución del 
problema adjunto de transporte permite el cálculo eficiente del gradiente 
funcional de la función objetivo, incluyendo el caso de condiciones de bordes 
de Fresnel.


\vspace{1cm}
\noindent
Palabras clave: 
Tomografía óptica,
Ecuación de transporte radiativo, 
Espectroscopia del infrarrojo cercano, 
Propagación de la radiación en la materia,
Problema inverso.
\pagestyle{empty}
