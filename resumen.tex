\pagestyle{empty}
\chapter*{}
\addcontentsline{toc}{chapter}{Resumen}

\begin{center}
\begin{large}
\textbf{Algoritmos eficientes para aplicaciones en tomografía óptica}
\end{large}
\end{center}

\vspace{1cm}
Esta Tesis presenta un algoritmo {\em paralelo} eficiente  
para la resolución del problema inverso en 
tomografía óptica basado en la ecuación de transferencia radiativa en el dominio temporal. 
La ecuación de transferencia radiativa provee un modelo 
físicamente preciso para el transporte de fotones 
en el tejido biológico, pero el alto costo computacional 
asociado a su resolución representa un obstáculo 
para su utilización en tomografía óptica en el dominio 
temporal, y otras áreas. En esta Tesis se aborda este problema 
mediante un número de innovaciones computacionales y de 
modelado, que incluyen 1) Una estrategia de paralelización 
basada en la descomposición del dominio espacial que presenta 
\textit{escalabilidad ideal} para los problemas directos 
e inversos; 2) Una estrategia de Fuentes Múltiples Simultaneas 
(FMS) que resuelve el problema inverso de transporte 
con un costo computacional que es {\em independiente del 
número de fuentes empleadas}, y el cual acelera significativamente la reconstrucción de los parámetros ópticos: 
se demuestra un factor de aceleración de seis en esta Tesis. 
Finalmente, esta contribución presenta 3) Una derivación 
intuitiva de la formulación del problema adjunto para la 
cálculo de los gradientes funcionales, que incorpora 
las condiciones de borde de Fresnel---generalizando 
resultados reportados en la literatura para las condiciones 
de borde de vacío. Se presentan 
soluciones de problemas inversos realistas en 2D, 
que fueron obtenidos en un cluster de computadoras con 
hasta 256 procesadores. La combinación de la estrategia 
de paralelización con la estrategia FMS redujo el tiempo 
computacional requerido para la resolución de los problemas inversos en varios órdenes de magnitud, de meses a unas pocas horas. 

\vspace{1cm}
\noindent
Palabras clave: 
Tomografía óptica,
Ecuación de transporte radiativo, 
Espectroscopia del infrarrojo cercano, 
Propagación de la radiación en la materia,
Problema inverso.
\pagestyle{empty}
