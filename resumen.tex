\chapter*{}%
\addcontentsline{toc}{chapter}{Resumen}%

\begin{center}
\begin{large}
\textbf{Algorítmos eficientes para la resolución 
del problema inverso en tomografía óptica}
\end{large}
\end{center}

\vspace{1cm}
En este trabajo se desarrollan   
algorítmos eficientes para la resolución de los problemas directos 
e inversos necesarios en la disciplina de tomografía óptica. Para ello, utilizamos la Ecuación de Transporte 
Radiativo (ETR) como modelo físico de transporte para la radiación 
en la materia. 
La ERT es resuelta por medio de algoritmos basados en el Método  
de Continuación de Fourier en Ordenadas Discretas (FC-DOM, de su sigla en inglés). Estos algorítmos 
permiten resolver la ERT de forma eficiente y en entornos de máquinas paralelas 
con escabilidad ideal, como se muestra en esta tésis. 

La identificación de una capa límite, previamente no descripta en la literatura, 
permite explicar por qué los métodos numéricos existentes para la resolución de la ERT 
no presentan un alto orden de convergencia. En esta tésis se presenta 
una teoría que permite identificar dicha capa límite, proveyendo  
estrategias para su resolución que logran alto órden de convergencia 
a la ERT, lo cual es demostrado en la sección~\ref{sec:blayer} para el caso de una única dimensión espacial.

Para la resolución del problema inverso en tomografía óptica utilizamos el método de minimización 
de Broyden–Fletcher–Goldfarb–Shanno (BFGS) con uso de memoria reducido 
(lm-BFGS, de su sigla en inglés), 
para el cual es necesario el cálculo eficiente del gradiente 
de la función objetivo involucrada. Para ello, desarrollamos la teoría 
que permite el cálculo eficiente de dicho gradiente  
mediante la resolución del problema adjunto. 


\vspace{1cm}
\noindent
Palabras clave: 
Tomografía óptica,
Ecuación de transporte radiativo, 
Espectroscopía del infrarrojo cercano, 
Propagación de la radiación en la materia,
Problema inverso.
