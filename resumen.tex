\pagestyle{empty}
\chapter*{}
\addcontentsline{toc}{chapter}{Resumen}
\vspace{-3cm}
\begin{center}
\begin{large}
\textbf{Algoritmos eficientes para aplicaciones en tomografía óptica}
\end{large}
\end{center}

\vspace{1cm}
En esta Tesis se presenta un algoritmo {\em paralelo} eficiente  
para la resolución del problema inverso en 
tomografía óptica basado en la ecuación de transferencia radiativa en el dominio temporal. 
Esta ecuación provee un modelo 
físicamente preciso para el transporte de fotones 
en el tejido biológico, pero el alto costo computacional 
asociado a su resolución representa un obstáculo 
para su utilización en tomografía óptica, y otras áreas. En esta Tesis se aborda este problema 
mediante un número de innovaciones computacionales y de 
modelado, que incluyen 1) La incorporación 
de un método espectral de alto orden (continuación de Fourier 
en ordenadas discretas (FC--DOM)) que permite resolver la 
ecuación de transporte con gran precisión y con reducido esfuerzo 
computacional.
2) Una estrategia de paralelización 
basada en la descomposición del dominio espacial que presenta 
\textit{escalabilidad ideal} para los problemas directos 
e inversos; 3) Una estrategia de Fuentes Múltiples Superpuestas 
(FMS) que resuelve el problema inverso de transporte 
con un costo computacional que es {\em independiente del 
número de fuentes empleadas}, y el cual acelera significativamente la reconstrucción de los parámetros ópticos.
Adicionalmente, esta contribución presenta una derivación 
intuitiva de la formulación del problema adjunto para el  
cálculo de los gradientes funcionales, que incorpora 
las condiciones de borde de Fresnel. Se presentan 
soluciones de problemas inversos realistas en 2D, 
que fueron obtenidos en un cluster de computadoras con 
hasta 256 procesadores. La combinación del método 
FC--DOM, la estrategia de paralelización y la técnica 
FMS redujo el tiempo 
computacional requerido para la resolución de estos problemas, 
de meses a unas pocas horas. 

\vspace{1cm}
\noindent
Palabras clave: 
Propagación de la radiación en la materia,
Tomografía óptica,
Ecuación de transporte radiativo, 
Espectroscopia del infrarrojo cercano, 
Problema inverso.
\pagestyle{empty}
