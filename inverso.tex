\chapter{El problema inverso}%
\lhead{\thepage}
\rhead{\textit{El problema inverso}} \\
\vspace{0.01\textheight}
\label{sec:inverso}
%\pagebreak

En las secciones previas abordamos el problema directo de transporte 
de radiación en la materia mediante la ETR. En síntesis, el problema 
directo consta de, dados los parámetros ópticos $a(\x)$, $b(\x)$, la función  
de fase $\eta(\hth\cdot \hth')$, la velocidad de la luz en el medio participante 
$c$, las fuentes internas $s$ y las condiciones iniciales y de contorno, 
encontrar la solución $\ut$ a la ecuación~\eqref{eq:RTE}.

Para el problema inverso

En la figura esquematizamos 
\section{Cálculo del gradiente}


\subsection{FC-DOM}
\section{Reconstrucciones numéricas}
\label{sec:inverseres}
\section{Consecuencias de la capa límite}
