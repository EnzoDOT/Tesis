\pagestyle{empty}
\chapter*{}
\addcontentsline{toc}{chapter}{Abstract}
\clearpage \pagebreak 
\begin{center}
\begin{large}
\textbf{Efficient algorithms for optical tomography applications}
\end{large}
\end{center}

\vspace{1cm}
This Thesis presents an efficient {\em parallel} radiative
  transfer--based inverse--problem solver for time--domain optical
  tomography. This equation provides a physically
  accurate model for the transport of photons in biological tissue, 
  but the high computational cost associated with its solution has
  hindered its use in time--domain optical--tomography and other
  areas. In this Thesis this problem is tackled by means of a number of 
  computational and modeling innovations, including 
  1) The incorporation of a high--order spectral method 
(Fourier continuation discrete ordinates method (FC--DOM)), 
allowing the solution of the transport equation with 
high precision and with low computational cost.  
  2) A spatial
  parallel--decomposition strategy with \textit{perfect parallel
    scaling} for the forward and inverse problems of optical
  tomography on parallel computer systems; and, 3) A Multiple
  Staggered Source method (MSS) that solves the inverse transport
  problem at a computational cost that is {\em independent of the
    number of sources employed}, and which significantly accelerates
  the reconstruction of the optical parameters. Additionally, this
  contribution presents an intuitive derivation of the
  adjoint--based formulation for evaluation of functional gradients,
  including general Fresnel boundary conditions. Solutions of large and realistic 2D inverse problems are
  presented, which were produced on a 256--core computer
  system. The combined FC--DOM/parallel/MSS acceleration approach reduced the required computing times by several orders of magnitude, from months to a few hours.


\vspace{1cm}
\noindent
Key words: 
Radiation transport through matter,
Optical tomography,
Radiative transfer equation, 
Near infrared spectroscopy, 
Inverse problem.
\pagestyle{empty}
