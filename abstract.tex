\pagestyle{empty}
\chapter*{}
\addcontentsline{toc}{chapter}{Abstract}

\begin{center}
\begin{large}
\textbf{Efficient algorithms for optical tomography applications}
\end{large}
\end{center}

\vspace{1cm}
In this work we develop efficient algorithms for the resolution 
of the direct and inverse problems in optical tomography. 
To this end, we utilize the Radiative Transfer Equation (RTE) 
as a physical model for the transport of photons in matter. 

The RTE is solved by means of algorithms based on the Fourier 
Continuation Method in conjunction with the Discrete Ordinates method (FC-DOM method).
These algorithms allows to solve the RTE in an efficient manner on parallel 
machines with ideal scalability, as it is shown in this thesis.

We also present an identification and characterization of boundary layer structures 
in the solutions of the RTE, providing a strategy for its resolution. 
The 
boundary layer theory provided strategies for the numerical resolution 
of the RTE with high order convergence in all the variables involved, 
as demonstrated in section~\ref{sec:blayer} in one spatial dimension.

For the inverse problem resolution in optical tomography, 
we employ the quasi-Newton Broyden–Fletcher–Goldfarb–Shanno method, 
with limited use of memory (lm-BFGS). An adjoint problem methodology 
is developed, which allows the efficient evaluation of the objective 
function involved, including Fresnel boundary conditions.


\vspace{1cm}
\noindent
Key words: 
Optical tomography,
Radiative transfer equation, 
Near infrared spectroscopy, 
Radiatition transport through matter,
Inverse problem.
