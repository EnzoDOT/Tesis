\pagestyle{empty}
\chapter*{}
\addcontentsline{toc}{chapter}{Abstract}
\clearpage \pagebreak 
\begin{center}
\begin{large}
\textbf{Efficient algorithms for optical tomography applications}
\end{large}
\end{center}

\vspace{1cm}
This Thesis presents an efficient {\em parallel} radiative
  transfer-based inverse-problem solver for time-domain optical
  tomography. The radiative transfer equation provides a physically
  accurate model for the transport of photons in biological tissue,
  but the high computational cost associated with its solution has
  hindered its use in time-domain optical-tomography and other
  areas. In this paper this problem is tackled by means of a number of
  computational and modeling innovations, including 1) A spatial
  parallel-decomposition strategy with \textit{perfect parallel
    scaling} for the forward and inverse problems of optical
  tomography on parallel computer systems; and, 2) A Multiple
  Staggered Source method (MSS) that solves the inverse transport
  problem at a computational cost that is {\em independent of the
    number of sources employed}, and which significantly accelerates
  the reconstruction of the optical parameters: a six-fold MSS
  acceleration factor is demonstrated in this paper. Finally, this
  contribution presents 3) An intuitive derivation of the
  adjoint-based formulation for evaluation of functional gradients,
  including general Fresnel boundary conditions---thus, in particular,
  generalizing results previously available for vacuum boundary
  conditions. Solutions of large and realistic 2D inverse problems are
  presented in this paper, which were produced on a 256-core computer
  system. The combined parallel/MSS acceleration approach reduced the
  required computing times by several orders of magnitude, from months
  to a few hours.


\vspace{1cm}
\noindent
Key words: 
Optical tomography,
Radiative transfer equation, 
Near infrared spectroscopy, 
Radiatition transport through matter,
Inverse problem.
\pagestyle{empty}
