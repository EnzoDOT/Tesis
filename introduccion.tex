%\documentclass[a4,14pt]{book}

\pagestyle{fancy}
\lhead{\thepage}
\rhead{\textit{Introducción}}

%\begin{document}


\chapter{Introducción}
\vspace{0.01\textheight}
\pagebreak


\section{Descripción general del trabajo}  

El objetivo principal planteado en esta Tesis es el desarrollo 
de estrategias computacionales para el 
estudio de problemas de transporte de partículas neutras, y para la 
caracterización de los medios atravesados por estas.
Específicamente, concierne al estudio de radiación de fotones y 
flujos de neutrones.

En concreto, lo que buscamos es resolver el llamado  
{\bf problema inverso}, que consiste en determinar las características 
del medio participante basados en los datos obtenidos por mediciones 
de radiación producida por fuentes externas conocidas.
La idea fundamental radica en suponer inicialmente unos parámetros 
físicos del medio (coeficientes de absorción y dispersión en distintas 
regiones del mismo), y luego, mediante la resolución del {\bf problema 
directo} (es decir, la solución de la ecuación de transporte), 
simular los resultados que se obtendrían en los detectores. 
Seguidamente, estos se contrastan con los valores reales experimentales, 
y mediante una serie de iteraciones se busca que las diferencias 
entre las predicciones teóricas y las mediciones experimentales 
se minimicen.
Si bien esto tiene numerosas aplicaciones, daremos prioridad al estudio 
orientado hacia la tomografía óptica, y en particular, a la 
caracterización de medios biológicos (tejido humano). 
Por ello, sin perder generalidad, nos referiremos mayormente 
durante el transcurso de esta Tesis, a la ecuación de transporte radiativo (ETR).

La resolución de este problema exige enormes recursos computacionales, 
fundamentalmente debido a su alta dimensionalidad.
Para el problema directo, deben tomarse en cuenta las tres dimensiones 
espaciales, y en cada punto se deben considerar las direcciones 
angulares para la propagación de la radiación. 
Sumado a esto y dado que su solución se obtiene a través de la 
evolución de un estado inicial, se debe incluir la variable temporal.
El problema directo por sí mismo presenta grandes complejidades 
numéricas, especialmente si los escenarios a resolver incluyen 
medios inhomogéneos, geometrías complicadas, y condiciones de 
borde específicas.
Por ello, es muy común el uso de la aproximación de difusión, 
que simplifica notablemente los cálculos. 
Desafortunadamente, esta aproximación no es válida para medios 
no dispersivos, especialmente para los casos tratados en nuestro trabajo.

El problema inverso es aún mas complicado. 
En este caso, se supone que se cuenta con resultados experimentales, 
por ejemplo, el flujo de radiación emitido por ciertas fuentes, 
y luego medido por ciertos detectores. 
Lo que se busca determinar, son las propiedades del medio.  
Para ello, las simulaciones numéricas parten de   
una configuración inicial, con la que se resuelve la ecuación de transporte, 
obteniendo los resultados teóricos que representan a las 
mediciones en los detectores. 
Este procedimiento se itera, modificando las propiedades del medio, hasta que las 
diferencias entre las predicciones teóricas y los valores experimentales 
medidos en los detectores sean despreciables.
Como se percibe, este problema exige numerosos pasos de iteración, 
donde en cada uno de ellos se resuelve un problema directo. 
Además, la minimización exige el almacenamiento en memoria de 
los diferentes resultados, que permitan inferir los cambios en 
los parámetros que llevan hacia el mínimo.

En nuestro trabajo aplicamos un novedoso método de resolución de 
ecuaciones diferenciales, para el tratamiento de la ecuación de 
transporte. Se trata del método de continuación de Fourier para 
ordenadas discretas (FC--DOM, por {\it Fourier Continuation -- Discrete 
Ordinates Method}), que explicaremos en detalle. 
Este método convierte cualquier función arbitraria en 
periódica, y por ende, permite la resolución de las ecuaciones 
utilizando transformadas de Fourier. Con ello se logran enormes 
precisiones y una gran economía de recursos computacionales.
Dedicamos un capítulo entero al desarrollo del método, a
sus aplicaciones en problemas modelo y a la descripción teórica de 
experimentos, análisis de errores y comparaciones con 
otros métodos.

Por supuesto que al tratarse de problemas de alta complejidad 
numérica, es menester dedicar esfuerzos en su resolución mediante 
técnicas computacionales de alta performance. 
Si bien es común encontrar que los problemas directos se resuelven 
utilizando arquitecturas gráficas (GPU), generalmente estos 
métodos se basan en la aproximación difusiva, la cual no es 
viable en problemas de tomografía óptica como los que discutiremos.
Tampoco son útiles estas arquitecturas para el tratamiento del 
problema inverso, debido a la enorme demanda de almacenamiento 
de memoria que requieren estos cálculos.
En nuestro trabajo, desarrollamos una estrategia de paralelización 
basada en una descomposición de dominio. Combinando esta estrategia 
con el método FC--DOM, se logra un factor de escalabilidad que supera 
al número de procesadores empleados (escalabilidad ideal).
Esta estrategia de paralelización se explicará en detalle, 
junto con ciertos ejemplos de su aplicación.

Un componente adicional que desarrollamos en nuestra investigación, 
consiste en el tratamiento de las fuentes, que pueden encenderse 
en diferentes tiempos, como un conjunto de fuentes generalizadas. 
Esto significa, que consideramos a varias fuentes independientes, 
como una fuente única que varía en el tiempo. 
De esta manera, es posible resolver el problema 
inverso una sóla vez, en lugar de hacerlo una vez por cada fuente. 
Con esta estrategia logramos reducir considerablemente los tiempos de 
cálculo, haciéndolos independientes del número de fuentes. 
Esto es importante, porque al incrementar las fuentes y los 
detectores, y al encender las primeras en distintos tiempos, 
se logra obtener mayor información del sistema, y por consiguiente, 
una mejor reconstrucción de las características del mismo.

Todas las estrategias y métodos de cálculo serán desarrolladas 
en detalle, brindando ejemplos de aplicaciones, y comparaciones 
con resultados analíticos y experimentales. 
En particular, reproduciremos dos ejemplos reales, de un cuello 
humano y de una cabeza humana, en los 
cuales se reconstruyen imágenes de resonancia magnética, a las 
que le agregamos algunas inclusiones, que pueden representar 
tumores o regiones de activación hemodinámica.
Mostraremos la excelente reproducción teórica de estas imágenes, 
junto a la discusión correspondiente que demuestra la enorme 
eficiencia de nuestros métodos.

La organización de esta Tesis es la siguiente: 
En el Capítulo~\ref{cap:forw} se describe la ETR y su interpretación 
física. Se presenta el método FC--DOM, y su validación mediante 
comparación con resultados analíticos y experimentales. 
Este capítulo finaliza con la identificación y 
caracterización de las estructuras de capa límite existentes 
en las soluciones a la ETR, donde se propone un método para 
la correcta resolución de estas estructuras. 
 En el Capítulo~\ref{cap:inverso}  se detallan los métodos 
y algoritmos utilizados para la resolución del problema inverso. 
Se realizan reconstrucciones de imágenes tomograficas en un cuello 
y cabeza humana, para la identificación de tumores y actividad hemodinámica. 
Finalmente, en el Capítulo~\ref{cap:conc} se sintetizan los principales 
logros de este trabajo.


%%%%%%%%%%%%%%%%%%%%%%%%%%%%%%%%%%%%%%%%%%%%%%%%%%%%%%%%%%%%%%%%%%%%%
\bigskip
\section{Motivación}

La importancia del modelado del transporte de partículas neutras 
difícilmente puede ser sobrestimada, ya que encuentra aplicaciones 
en diversas áreas de la ciencia y la tecnología, 
como por ejemplo el transporte de radiación 
 térmica para aplicaciones industriales~\cite{Howell2010, Thynell1998}, 
 la dinámica de gases~\cite{Duderstadt1979}, 
 el transporte de radiación en atmósferas estelares y 
 planetarias~\cite{Qin2015, Dymond1997, Chandrasekhar1960}, 
 el diagnóstico médico de tumores~\cite{Zhu2005, Zhu2010, Fujii2016b}, 
 la planificación y dosificación de radiación en radioterapia~\cite{Vassiliev2010,Bedford2019}, 
 el diagnóstico de artritis~\cite{Klose2002, Netz2001}, 
 la tomografía óptica por fluorescencia~\cite{Klose2005,Klose2010, Ren2010},
 y el modelado de transporte de neutrones para el desarrollo 
 y diseño de reactores nucleares~\cite{Larsen2006, Sanchez1982, Anli2006}, 
 entre otros~\cite {Mishchenko1999, Prasher2003}. 
  
 La tomografía óptica es una técnica no invasiva en la que
 radiación electromagnética es inyectada dentro del medio
 en estudio.
 La radiación emergente es detectada mediante fotodectectores 
 ubicados en el contorno del medio que se está analizando. 
 A partir de la luz detectada, se pueden reconstruir
 los parámetros ópticos que caracterizan al mismo.
 En las aplicaciones médicas, esta técnica ha demostrado numerosas 
 ventajas, entre ellas su bajo costo, la fácil portabilidad, y 
 el empleo de radiación no ionizante (no es cancerígena). 
 Se utiliza en general radiación proveniente de láseres, en un 
 rango del infrarrojo cercano, entre 600 a 900 nanómetros, que 
 corresponde a la ventana de penetración de la piel humana.
 Para estas longitudes de onda, el tejido 
 biológico se comporta como un medio altamente dispersivo, y 
 permite sensar varios centímetros en el interior del mismo~\cite{Boas2001}.
Los fotones viajan a través del tejido, sufriendo múltiples colisiones 
 de dispersión elástica, describiendo trayectorias aleatorias.
 Los componentes del tejido biológico pueden identificarse y caracterizarse
 mediante el denominado coeficiente de absorción óptica.
 
La tomografía óptica sirve para obtener información tanto anatómica como funcional.
La reconstrucción de las propiedades de absorción en el tejido humano 
 permiten la identificación de tumores~\cite{Zhu2005, Zhu2010, Fujii2016b}.
Estas técnicas se basan en la detección de hemoglobina, que puede hacerse 
favorablemente utilizando radiación del infrarrojo cercano.
Las células tumorales generan ---mediante el proceso de angiogénesis---, 
los vasos sanguíneos necesarios para su alimentación y reproducción.
La obtención de imágenes funcionales del cerebro~\cite{Boas2001, bluestone2001, Arridge1999}, se basa en los grandes cambios 
que sufre el coeficiente de absorción de la hemoglobina 
alrededor de los $650$ nm, cuando esta absorbe oxígeno 
(hemoglobina oxigenada vs.~desoxihemoglobina). 
 La activación de diferentes regiones del cerebro 
 es acompañada por una respuesta hemodinámica que lleva sangre oxigenada 
 a las mismas, lo que permite detectar que regiones están funcionando 
 ante determinadas actividades. 
 
 
 
\pagestyle{empty}

%\end{document}


