\chapter{Introducción}
\lhead{\thepage}
\rhead{\textit{Introducción}}
\vspace{0.01\textheight}
\pagebreak
La ecuación de transporte radiativo (ETR de ahora en adelante) es la forma lineal de la ecuación de transporte de Boltzmann, 
cuyo dominio de definición es el espacio de las fases. Ésta ecuación modela el transporte 
de radiación electromagnética en un medio participante ---la radiación en el proceso de transporte 
interactúa con el medio, siendo absorbida, dispersada y emitida por éste---. La ETR modela también el transporte de partículas neutras en
 general (para las cuales las interacciones mutuas entre las partículas involucradas 
 en el proceso de transporte puede ser despreciada). La importancia del modelado del transporte de 
 partículas neutras (siendo los fotones la partícula involucrada en el proceso de transporte de 
 radiación electromagnética) difícilmente puede ser sobrestimada, ya que dicho modelado encuentra aplicaciones en diversas
 áreas de la ciencia y la tecnología, como lo es el transporte de radiación 
 térmica para aplicaciones industriales~\cite{Howell2010, Thynell1998}, la dinámica de 
 gases~\cite{Duderstadt1979}, el transporte de radiación en atmósferas estelares y 
 planetarias~\cite{Qin2015, Dymond1997, Chandrasekhar1960}, el diagnóstico médico de 
 tumores~\cite{Zhu2005, Zhu2010, Fujii2016b}, la planificación y dosificación 
 de radiación en radioterapia~\cite{Vassiliev2010,Bedford2019}, el diagnóstico de artritis~\cite{Klose2002, Netz2001} 
 y el modelado de transporte de neutrones para el desarrollo 
 y diseño de reactores nucleares~\cite{Larsen2006, Sanchez1982, Anli2006}, entre otros~\cite {Mishchenko1999, Prasher2003}. 
 En cuanto al transporte de radiación electromagnética, la ETR puede derivarse, bajo ciertas aproximaciones, como un límite asintótico 
 para soluciones de las ecuaciones de Maxwell. La cantidad modelada por la ETR, comúnmente denominada 
 intensidad específica, queda relacionada en dicho límite con el vector de Poynting~\cite{Mishchenko2002, Ripoll2011} 
 en la teoría ondulatoria del electromagnetismo. Los fenómenos ondulatorios como la interferencia y la difracción no son
 capturado por la ETR, la cual expresa esencialmente la conservación de la
 energía irradiada a escala mesoscópica en el campo electromagnético.

 La tomografía óptica es una técnica tomográfica no invasiva en la que
 radiación electromagnética no ionizante es inyectada dentro del tejido en estudio.
 La radiación emergente es detectada mediante fotodectectores 
 ubicados en el contorno del tejido biológico que se está analizando. A
  partir de la luz detectada, el objetivo es la reconstrucción de
 los parámetros ópticos que caracterizan al tejido biológico. Esta técnica 
 tiene la capacidad de proporcionar información funcional y anatómica. 
 
 Algunas de las ventajas relativas de esta técnica tomográfica que se encuentra aún en desarrollo, respecto a 
 técnicas ya establecidas, son la portabilidad y su bajo costo. 
 La utilización de radiación no 
 ionizante implica que este tipo de radiación no es cancerígena, a diferencia de los rayos-X, 
 por lo cual es una técnica que podría reemplazar a las tomografías 
 computarizadas de rayos-X para el diagnóstico de cáncer de mama, entre otros. 
 
 La ETR es utilizada 
 en tomografía óptica como
 un modelo para el transporte de fotones en el tejido biológico a una longitud de onda previamente determinada, 
 que típicamente proviene de una fuente láser.
 La ``ventana óptica'' ubicada en el rango de longitudes de onda del espectro electromagnético comprendida entre los 600 a 900 nanómetros permite a la radiación en el 
 infrarrojo cercano penetrar y sensar varios centímetros en el interior del tejido biológico~\cite{Boas2001}. 
 Para estas longitudes de onda del espectro electromagnético, el tejido humano se comporta 
 como un medio altamente dispersivo. Los fotones viajan a través del
 tejido, sufriendo múltiples colisiones de dispersión elástica, describiendo trayectorias 
aleatorias.

 Los componentes del tejido biológico pueden identificarse y caracterizarse
 explotando el denominado coeficiente de absorción óptica, o en conjunto 
 los coeficientes de absorción y de dispersión, ambos dependientes 
 de los componentes del tejido.
 Debido a las interacciones altamente dispersivas dentro del tejido,
 la información de la trayectoria de los fotones se pierde, y no puede ser efectivamente explotada,
 lo que da como resultado los bajos niveles de resolución que se obtienen en esta disciplina. 
 Esta es una característica particular de la técnica, que la distingue de la tomografía 
 por rayos-X, donde los fotones sufren poca dispersión y la transformada de 
Radon resulta eficiente para la obtención de las imágenes. A sí mismo, 
 dado que el coeficiente de absorción depende fuertemente en la longitud de onda 
 de la radiación electromagnética utilizada, esta técnica provee alto contraste, 
 ya que dicha longitud de onda se puede ajustar a la longitud de onda de absorción 
 del medio de interés que quiere detectarse, \eg utilizando radiación alrededor de los $650$nm 
 para distinguir hemoglobina oxigenada de la desoxihemoglobina~\cite{Boas2001}. La angiogenesis 
 cumple un rol central en el crecimiento de los tumores. Este es 
 el mecanismo mediante el cual las células tumorales generan los vasos sanguíneos 
 necesario para que las células cancerosas se alimenten y se reproduzcan. Por este 
 motivo un caso de particular interés es la detección de la hemoglobina, ya que esta indica la presencia de vasos sanguíneos, los cuales pueden ser un indicador de la presencia y del estado de un tumor, para su detección o para monitoreo y seguimiento en su tratamiento. 
 Por otra parte, 
 la activación de diferentes regiones del cerebro 
 es acompañada por una respuesta hemodinámica que lleva sangre oxigenada 
 a las mismas. La detección de la hemoglobina 
 oxigenada es por lo tanto un indicador de actividad cerebral, con aplicaciones 
 en neurociencias. 
 
 La reconstrucción de las propiedades de absorción en el tejido humano permiten la identificación de tumores~\cite{Zhu2005, Zhu2010, Fujii2016b},
 la obtención de imágenes funcionales del cerebro~\cite{Boas2001, bluestone2001, Arridge1999}, y la caracterización de los diferentes
 constituyentes del tejido humano para diagnóstico médico, entre otros. En este trabajo nos enfocamos en
 la reconstrucción del coeficiente de absorción, aunque
 el enfoque propuesto puede ser fácilmente generalizado para la reconstrucción 
 de otros parámetros, como \eg, el problema de determinar las fuentes en la ETR,
 que encuentra aplicaciones en la disciplina relacionada de
 tomografía óptica por fluorescencia~\cite{Klose2005,Klose2010, Ren2010}, y la reconstrucción 
 simultánea de los coeficientes de absorción y de dispersión~\cite{Ren2006,Prieto2017}.

 En este trabajo resolvemos el problema inverso en tomografía óptica como un problema 
 de minimización no lineal para el que empleamos
 el método cuasi-Newton de descenso de gradiente denominado 
 Broyden–Fletcher–Goldfarb–Shanno (BFGS) con uso de memoria reducido lm-BFGS~\cite{Byrd1995}. Éste es un método de minimización iterativo 
 que se vale del gradiente, y de una aproximación 
 al Hessiano de la función objetivo (a ser 
 introducida en la sección~\ref{sec:inverso}) para encontrar el mínimo deseado. En las referencias~\cite{Prieto2017,Boulanger2005} 
 el problema inverso en tomografía óptica en dos dimensiones espaciales (2D)
es presentado, utilizando la ETR dependiente del tiempo, sin considerar las condiciones de contorno de Fresnel. Debido a los altos costos computacionales,
 solo se emplean grillas numéricas de poca resolución,
 haciendo que estos enfoques no sean adecuados para situaciones reales, que pueden requerir
 de soluciones numéricas mejor convergidas, y dominios espaciales más grandes. 
 Los métodos de alto orden
 permiten obtener soluciones precisas de la ETR
 utilizando grillas numéricas mas gruesas, en comparación a métodos de menor orden. Esto impacta en los recursos computacionales requeridos para una precisión dada, en términos de tiempo computacional y exigencias de memoria, que se vuelven prohibitivos muy rápidamente debido a la alta dimensionalidad de la ETR. Los algoritmos propuestos 
 en este trabajo pueden ser fácilmente extendidos a geometrías de tres dimensiones espaciales (3D). En geometrías 2D, la ETR
 conserva la complejidad matemática de los problemas 3D, 
 exigiendo menor cantidad de recursos computacionales.

La ecuación de transporte radiativo proporciona un
modelo físicamente preciso para el transporte de fotones en el tejido biológico
\cite{Klose2009, Arridge2009}, pero su aplicabilidad en el contexto de
los problemas inversos en tomografía óptica se ha visto obstaculizada por el alto costo computacional requerido para su solución. 

El desarrollo de estrategias eficientes y de algoritmos capaces de distribuir el trabajo en máquinas paralelas es particularmente importante para la
resolución de los problemas directos e inversos para la ETR en 3D, donde las exigencias 
de memoria RAM y
los requisitos computacionales se vuelven prohibitivos muy rápidamente
debido a la alta dimensionalidad de la ETR, que en problemas 3D
involucra dos variables direccionales asociadas a las velocidades en 3D (definidas 
en la esfera unitaria), además de las tres variables espaciales y el tiempo.

En esta tesis se abordan estas dificultades y exigencias computacionales 
mediante una combinación de tres estrategias principales, (i)~el uso de
un enfoque espectral basado en el método de continuación de Fourier 
en ordenadas discretas
(FC-DOM)~\cite{Gaggioli2019} para la solución del 
problema de transporte de los fotones en el tejido humano; (ii)~Una implementación 
en máquinas paralelas sumamente eficiente del
Método FC-DOM basado en una estrategia de descomposición del dominio espacial 
involucrado; y (iii)~Una configuración de fuentes múltiples superpuestas (FMS), que utiliza ciertas combinaciones de fuentes que operan simultáneamente en lugar de
secuencias de fuentes únicas utilizadas en enfoques anteriores. En suma, 
dicha estrategia 
implementada en un clúster de computadoras con 256 núcleos físicos, da como resultado una aceleración del tiempo de cálculo de varios órdenes de
magnitud---lo que hace posible resolver problemas inversos en dominios del 
tamaño necesario para abordar 
problemas tales como el de la obtención de imágenes dentro de un modelo de cuello
humano simplificado considerado en la Sección~\ref{sec:inverseres}.

La estrategia de paralelización propuesta presenta una serie de
ventajas. A diferencia de otros algoritmos reportados en la literatura, el enfoque
presentado en este trabajo es altamente eficiente independientemente de
el número de fuentes empleadas (\cf \cite{Hielscher2004}).
Estrategias de computo en paralelo basadas en arquitecturas de GPU~\cite {Doulgerakis2017} han mostrado ser apropiadas para aplicaciones de tomografía óptica basadas en la aproximación de difusión. Desafortunadamente, 
la aproximación de difusión no es físicamente precisa en un amplio rango de situaciones,
y, debido a las altas exigencias en memoria de almacenamiento requerida para 
la resolución del problema inverso ---el cual exige guardar en memoria soluciones completas 
de la ETR para el problema 
directo y el problema adjunto--- la utilización de GPUs en este contexto no parece viable. 
Cabe mencionar que el uso de GPUs es una estrategia valida para reconstrucciones 
basadas en métodos estocásticos de Monte Carlo. Pero estos métodos son intrínsecamente 
altamente ineficientes. En la 
referencia~\cite{Coelho2014} se realiza una revisión de algoritmos propuestos en la literatura 
para la paralelización de la ETR. Todas las estrategias de paralelización muestran una eficiencia significativamente por debajo de la ideal, con excepción de la ref.~\cite {Colomer2013}, la cual presenta una estrategia de paralelización 
por encima de la ideal, pero
la aplicabilidad del método está restringida a medios no absorbentes y no dispersantes, 
para los cuales se conocen soluciones analíticas. La estrategia de paralelización 
desarrollada en esta tesis presenta escalabilidad ideal, independientemente 
del régimen de transporte en el que se requiere resolver la ETR, sin restricciones 
para los coeficientes de absorción y dispersión empleados, el número de fuentes o el número de ordenadas discretas que necesiten utilizarse. Como referencia, se obtuvo una eficiencia de 136,7 \% para las pruebas de escalabilidad realizadas con hasta 256 procesadores por
medio de la paralelización propuesta para el método FC-DOM (ver sec.~\ref{subsec:FC-DOM} fig.~\ref{fig:scala}). En la referencia~\cite [p. 153]{Fujii2014} 
se informa un tiempo de cálculo de 44,3 horas para la resolución de la ETR de un problema modelo 
en 2D por medio un único procesador. La solución del mismo problema utilizando 
los mismos parámetros y la misma resolución de la grilla numérica, corriendo en 64 procesadores, se obtiene mediante la estrategia de paralelización del algoritmo FC-DOM propuesto en menos de treinta minutos.

Como se mencionó anteriormente, una reducción adicional significativa en el
el tiempo de cálculo requerido para la solución del problema inverso es
lograda mediante la explotación del método FMS propuesto---el cual, combinando
múltiples fuentes en cada solución a la ETR, reduce el número de
de soluciones ETR directas y adjuntas requeridas. 
En esta tesis se demuestra la aceleración por un factor de seis, 
para la precisión en la reconstrucción del problema inverso,
relativo al tiempo requerido por el ``método de barrido'' 
utilizado de manera ubicua en tomografía óptica.

