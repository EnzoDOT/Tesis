%\documentclass[a4,14pt]{book}

\pagestyle{fancy}
\lhead{\thepage}
\rhead{\textit{Introducción}}

%\begin{document}


\chapter{Introducción}
\vspace{0.01\textheight}
\pagebreak


\section{Descripción general del trabajo}  

El objetivo principal planteado en esta Tesis es el desarrollo 
de estrategias computacionales para el 
estudio de problemas de transporte de partículas neutras, y para la 
caracterización de los medios atravesados por estas.
Específicamente, concierne al estudio de radiación de fotones y 
flujos de neutrones.

En concreto, lo que buscamos es resolver el llamado  
{\bf problema inverso}, que consiste en determinar las características 
del medio participante basados en los datos obtenidos por mediciones 
de radiación producida por fuentes externas conocidas.
La idea fundamental radica en suponer inicialmente unos parámetros 
físicos del medio (coeficientes de absorción y dispersión en distintas 
regiones del mismo), y luego, mediante la resolución del {\bf problema 
directo} (es decir, la solución de la ecuación de transporte), 
simular los resultados que se obtendrían en los fotodetectores. 
Seguidamente, estos se contrastan con los valores reales experimentales, 
y mediante una serie de iteraciones se busca que las diferencias 
entre las predicciones teóricas y las mediciones experimentales 
se minimicen.
Si bien esto tiene numerosas aplicaciones, daremos prioridad al estudio 
orientado hacia la tomografía óptica, y en particular, a la 
caracterización de medios biológicos (tejido humano). 
Por ello, sin perder generalidad, nos referiremos mayormente 
durante el transcurso de esta Tesis, a la ecuación de transporte radiativo (ETR).

La resolución de este problema exige enormes recursos computacionales, 
fundamentalmente debido a su alta dimensionalidad.
Para el problema directo, deben tomarse en cuenta las tres dimensiones 
espaciales, y en cada punto se deben considerar las direcciones 
angulares para la propagación de la radiación. 
Sumado a esto y dado que su solución se obtiene a través de la 
evolución de un estado inicial, se debe incluir la variable temporal.
El problema directo por sí mismo presenta grandes complejidades 
numéricas, especialmente si los escenarios a resolver incluyen 
medios inhomogéneos, geometrías complicadas, y condiciones de 
borde específicas.
Por ello, es muy común el uso de la aproximación de difusión, 
que simplifica notablemente los cálculos. 
Desafortunadamente, esta aproximación no es válida para medios 
no dispersivos, especialmente para los casos tratados en nuestro trabajo.

El problema inverso es aún mas complicado. 
En este caso, se supone que se cuenta con resultados experimentales, 
por ejemplo, el flujo de radiación emitido por ciertas fuentes, 
y luego medido por ciertos detectores. 
Lo que se busca determinar, son las propiedades del medio.  
Para ello, las simulaciones numéricas parten de   
una configuración inicial, con la que se resuelve la ecuación de transporte, 
obteniendo los resultados teóricos que representan a las 
mediciones en los detectores. 
Este procedimiento se itera, modificando las propiedades del medio, hasta que las 
diferencias entre las predicciones teóricas y los valores experimentales 
medidos en los detectores sean despreciables.
Este problema exige numerosos pasos de iteración, 
donde en cada uno de ellos se resuelve un problema directo. 
Además, la minimización exige el almacenamiento en memoria de 
los diferentes resultados, que permitan inferir los cambios en 
los parámetros que llevan hacia el mínimo.

En nuestro trabajo aplicamos un novedoso método de resolución de 
ecuaciones diferenciales, para el tratamiento de la ecuación de 
transporte. Se trata del método de continuación de Fourier para 
ordenadas discretas (FC--DOM, por {\it Fourier Continuation -- Discrete 
Ordinates Method}), que explicaremos en detalle. 
Este método convierte cualquier función arbitraria en 
periódica, y por ende, permite la resolución de las ecuaciones 
utilizando transformadas de Fourier. Con ello se logran resultados 
precisos y economía de recursos computacionales. Esto es de enorme importancia 
en la disciplina de tomografía óptica, debido a sus demandas computacionales. 
Dedicamos un capítulo entero al desarrollo del método, a
sus aplicaciones en problemas modelo y a la descripción teórica de 
experimentos, análisis de errores y comparaciones con 
otros métodos.

Por supuesto que al tratarse de problemas de alta complejidad 
numérica, es menester dedicar esfuerzos en su resolución mediante 
técnicas computacionales de alta performance. 

En nuestro trabajo, desarrollamos una estrategia de paralelización 
basada en una descomposición de dominio. Con ella, se logran factores 
de escalabilidad que superan al 
número de procesadores utilizados (escalabilidad ideal), y que resultan de combinar la estrategia  de descomposición de dominio 
con el método FC--DOM.
Esta estrategia de paralelización se explicará en detalle, 
junto con ciertos ejemplos de su aplicación.

Un componente adicional que desarrollamos en nuestra investigación, 
consiste en el tratamiento de las fuentes, que pueden encenderse 
en diferentes tiempos, como un conjunto de fuentes generalizadas. 
Esto significa, que consideramos a varias fuentes independientes, 
como una fuente única que varía en el tiempo. 
De esta manera, es posible resolver el problema 
inverso una sóla vez, en lugar de hacerlo una vez por cada fuente. 
Con esta estrategia logramos reducir considerablemente los tiempos de 
cálculo, haciéndolos independientes del número de fuentes. 
Esto es importante, porque al incrementar las fuentes y los 
detectores, y al encender las primeras en distintos tiempos, 
se logra obtener mayor información del sistema, y por consiguiente, 
una mejor reconstrucción de las características del mismo.

Todas las estrategias y métodos de cálculo serán desarrolladas 
en detalle, brindando ejemplos de aplicaciones, y comparaciones 
con resultados analíticos y experimentales. 
En particular, reproduciremos dos ejemplos reales, de un cuello 
humano y de una cabeza humana, en los 
cuales se reconstruyen imágenes de resonancia magnética, a las 
que le agregamos algunas inclusiones, que pueden representar 
tumores o regiones de activación hemodinámica.
Mostraremos la excelente reproducción teórica de estas imágenes, 
junto a la discusión correspondiente que demuestra la enorme 
eficiencia de nuestros métodos.

El resto de esta Tesis se organiza como sigue. 
En la sección {\em Contexto general y motivación} del presente capítulo 
se presentan algunos aspectos 
generales de la teoría de transporte. Se incluye
 una revisión de bibliografía relevante, 
y detalles adicionales sobre las contribuciones realizadas en el marco de este trabajo.
En el Capítulo~\ref{cap:forw} se presenta 
la ETR, la cual es utilizada en el contexto 
de esta Tesis como modelo directo 
en el problema inverso de tomografía óptica. Se desarrollan 
los métodos numéricos y algoritmos elaborados en esta Tesis 
para la resolución de la ETR, así como su validación 
mediante comparación con mediciones experimentales, y soluciones analíticas. Este capítulo finaliza con la identificación y 
caracterización de las estructuras de capa límite existentes 
en las soluciones a la ETR, donde se propone un método para 
la correcta resolución de estas estructuras. 
En el Capítulo~\ref{cap:inverso} se presenta el problema inverso 
en tomografía óptica. En este capítulo se detallan los métodos 
y algoritmos utilizados para la resolución del problema inverso. 
Se describen el ``método de barrido'' y la estrategia de Fuentes Múltiples Superpuestas, propuesta en esta Tesis. 
Se presenta una formulación para la obtención del gradiente 
funcional de la función objetivo mediante la utilización del método adjunto. El capítulo termina con dos 
reconstrucciones obtenidas por ambos métodos, donde se demuestra 
la eficiencia del enfoque propuesto. Finalmente, en el Capítulo~\ref{cap:conc} se sintetizan las conclusiones de este trabajo.


%%%%%%%%%%%%%%%%%%%%%%%%%%%%%%%%%%%%%%%%%%%%%%%%%%%%%%%%%%%%%%%%%%%%%
\bigskip
\section{Contexto general y motivación}

La ecuación de transporte radiativo ETR es la forma lineal de la ecuación de transporte de Boltzmann, 
cuyo dominio de definición es el espacio de las fases. Ésta ecuación modela el transporte 
de radiación electromagnética en un medio participante ---la radiación en el proceso de transporte 
interactúa con el medio, siendo absorbida, dispersada y emitida por éste---. La ETR modela también el transporte de partículas neutras en
 general (para las cuales las interacciones mutuas entre las partículas involucradas 
 en el proceso de transporte puede ser despreciada).

La importancia del modelado del transporte de partículas neutras 
difícilmente puede ser sobrestimada, ya que encuentra aplicaciones 
en diversas áreas de la ciencia y la tecnología, 
como por ejemplo el transporte de radiación 
 térmica para aplicaciones industriales~\cite{Howell2010, Thynell1998}, 
 la dinámica de gases~\cite{Duderstadt1979}, 
 el transporte de radiación en atmósferas estelares y 
 planetarias~\cite{Qin2015, Dymond1997, Chandrasekhar1960}, 
 el diagnóstico médico de tumores~\cite{Zhu2005, Zhu2010, Fujii2016b}, 
 la planificación y dosificación de radiación en radioterapia~\cite{Vassiliev2010,Bedford2019}, 
 el diagnóstico de artritis~\cite{Klose2002, Netz2001}, 
 la tomografía óptica por fluorescencia~\cite{Klose2005,Klose2010, Ren2010},
 y el modelado de transporte de neutrones para el desarrollo 
 y diseño de reactores nucleares~\cite{Larsen2006, Sanchez1982, Anli2006}, 
 entre otros~\cite {Mishchenko1999, Prasher2003}. 
 La ETR puede derivarse, bajo ciertas aproximaciones, como un límite asintótico 
 para soluciones de las ecuaciones de Maxwell. La cantidad modelada por la ETR, comúnmente denominada 
 intensidad específica, queda relacionada en dicho límite con el vector de Poynting~\cite{Mishchenko2002, Ripoll2011} 
 en la teoría ondulatoria del electromagnetismo. Los fenómenos ondulatorios como la interferencia y la difracción no son
 capturados por la ETR, la cual expresa esencialmente la conservación de la
 energía irradiada a escala mesoscópica en el campo electromagnético.

  
 La tomografía óptica es una técnica no invasiva en la que
 radiación electromagnética es inyectada dentro del medio
 en estudio.
 La radiación emergente es detectada mediante fotodectectores 
 ubicados en el contorno del medio que se está analizando. 
 A partir de la luz detectada, el objetivo es la reconstrucción de
 los parámetros ópticos que lo caracterizan. 
Esta técnica posee numerosas 
 ventajas relativas, entre las que se destacan su bajo costo, la fácil portabilidad, y 
 el empleo de radiación no ionizante (no cancerígena). 
 
La ETR es utilizada 
 en tomografía óptica como
 un modelo para el transporte de fotones a una longitud de onda prescripta, 
 típicamente proveniente de fuentes láser.
 La ``ventana óptica'' ubicada en el rango de longitudes de onda del espectro electromagnético comprendida entre los 600 a 900 nanómetros permite a la radiación en el 
 infrarrojo cercano penetrar y sensar varios centímetros en el interior del tejido biológico~\cite{Boas2001}. 
 Para estas longitudes de onda, el tejido 
 biológico se comporta como un medio altamente dispersivo, y 
 permite sensar varios centímetros en el interior del mismo~\cite{Boas2001}.
Los fotones viajan a través del tejido, sufriendo múltiples colisiones 
 de dispersión elástica, describiendo trayectorias aleatorias.
 Los componentes del tejido biológico pueden identificarse y caracterizarse
 mediante el denominado coeficiente de absorción óptica,  dependiente 
 de los componentes presentes en el medio.
 
La tomografía óptica sirve para obtener información tanto anatómica como funcional.
La reconstrucción de las propiedades de absorción en el tejido humano 
 permiten la identificación de tumores~\cite{Zhu2005, Zhu2010, Fujii2016b}.
Estas técnicas se basan en la detección de hemoglobina, que puede hacerse 
favorablemente utilizando radiación del infrarrojo cercano.
Las células tumorales generan -- mediante el proceso de angiogénesis --, 
los vasos sanguíneos necesarios para su alimentación y reproducción.
La obtención de imágenes funcionales del cerebro~\cite{Boas2001, bluestone2001, Arridge1999}, se basa en los grandes cambios 
que sufre el coeficiente de absorción de la hemoglobina 
alrededor de los $650$ nm, cuando esta absorbe oxígeno 
(hemoglobina oxigenada vs.~desoxihemoglobina). 
 La activación de diferentes regiones del cerebro 
 es acompañada por una respuesta hemodinámica que lleva sangre oxigenada 
 a las mismas, lo que permite detectar qué regiones están funcionando 
 ante determinadas actividades. 

La ETR proporciona un
modelo físicamente preciso para el transporte de fotones en el tejido biológico
\cite{Klose2009, Arridge2009}, pero su aplicabilidad en el contexto de
los problemas inversos en tomografía óptica se ha visto obstaculizada por el alto costo computacional requerido para su solución. 

El desarrollo de estrategias eficientes y de algoritmos capaces de distribuir el trabajo en máquinas paralelas es particularmente importante para la
resolución de los problemas directos e inversos para la ETR en 3D, donde las exigencias 
de memoria RAM y
los requisitos computacionales se vuelven prohibitivos muy rápidamente
debido a la alta dimensionalidad de la ETR, que en problemas 3D
involucra dos variables direccionales asociadas a las velocidades en 3D (definidas 
en la esfera unitaria), además de las tres variables espaciales y el tiempo. 

 En esta Tesis se presenta una descripción detallada 
de capas límite exponenciales presentes en la solución de la ETR, previamente no discutidas ni explotadas en la literatura. Como se demuestra en la Sección~\ref{sec:blayer}, la correcta resolución de dichas capas límite 
permite el desarrollo de métodos numéricos sumamente eficientes para la resolución de la ETR. 

Abordamos las dificultades y exigencias computacionales para la resolución del problema inverso en tomografía óptica 
mediante una combinación de tres estrategias principales, (i)~el uso de
un enfoque espectral basado en el método de continuación de Fourier 
en ordenadas discretas
(FC-DOM)~\cite{Gaggioli2019} para la solución del 
problema de transporte de los fotones en el tejido humano; (ii)~Una implementación 
en máquinas paralelas sumamente eficiente del
Método FC-DOM basado en una estrategia de descomposición del dominio espacial 
involucrado; y (iii)~Una configuración de fuentes múltiples superpuestas (FMS), que utiliza ciertas combinaciones de fuentes que operan simultáneamente en lugar de
secuencias de fuentes únicas utilizadas en enfoques anteriores. En suma, 
dicha estrategia 
implementada en un clúster de computadoras con 256 núcleos físicos, da como resultado una aceleración del tiempo de cálculo de varios órdenes de
magnitud---lo que hace posible resolver problemas inversos en dominios del 
tamaño necesario para abordar 
problemas tales como el de la obtención de imágenes dentro de un modelo de cuello
humano considerado en la Sección~\ref{sec:inverseres}.

La estrategia de paralelización propuesta presenta una serie de
ventajas. A diferencia de otros algoritmos reportados en la literatura, el enfoque
presentado en este trabajo es altamente eficiente independientemente de
el número de fuentes empleadas (\cf \cite{Hielscher2004}).
Estrategias de computo en paralelo basadas en arquitecturas de GPU~\cite {Doulgerakis2017} han mostrado ser apropiadas para aplicaciones de tomografía óptica basadas en la aproximación de difusión. Desafortunadamente, 
la aproximación de difusión no es físicamente precisa en un amplio rango de situaciones,
y, debido a las altas exigencias en memoria de almacenamiento requerida para 
la resolución del problema inverso ---el cual exige guardar en memoria soluciones completas 
de la ETR para el problema 
directo y el problema adjunto--- la utilización de GPUs en este contexto no parece viable. 
Cabe mencionar que el uso de GPUs es una estrategia valida para reconstrucciones 
basadas en métodos estocásticos de Monte Carlo. Pero estos métodos son intrínsecamente 
altamente ineficientes. En la 
referencia~\cite{Coelho2014} se realiza una revisión de algoritmos propuestos en la literatura 
para la paralelización de la ETR. Todas las estrategias de paralelización muestran una eficiencia significativamente por debajo de la ideal, con excepción de la ref.~\cite {Colomer2013}, la cual presenta una estrategia de paralelización 
por encima de la ideal, pero
la aplicabilidad del método está restringida a medios no absorbentes y no dispersantes, 
para los cuales se conocen soluciones analíticas. La estrategia de paralelización 
desarrollada en esta Tesis presenta escalabilidad ideal, independientemente 
del régimen de transporte en el que se requiere resolver la ETR, sin restricciones 
para los coeficientes de absorción y dispersión empleados, el número de fuentes o el número de ordenadas discretas que necesiten utilizarse. Como referencia, se obtuvo una eficiencia de 136,7 \% para las pruebas de escalabilidad realizadas con hasta 256 procesadores por
medio de la paralelización propuesta para el método FC-DOM (ver sec.~\ref{subsec:FC-DOM} fig.~\ref{fig:scala}). En la referencia~\cite [p. 153]{Fujii2014} 
se informa un tiempo de cálculo de 44,3 horas para la resolución de la ETR de un problema modelo 
en 2D por medio un único procesador. La solución del mismo problema utilizando 
los mismos parámetros y la misma resolución de la grilla numérica, corriendo en 64 procesadores, se obtiene mediante la estrategia de paralelización del algoritmo FC-DOM propuesto en menos de treinta minutos.

Se logra una reducción adicional significativa en el
el tiempo de cálculo requerido para la solución del problema inverso mediante el método FMS propuesto---el cual, combinando
múltiples fuentes en cada solución a la ETR, reduce el número de
de soluciones ETR directas y adjuntas requeridas. 
En esta Tesis se demuestra la aceleración por un factor de seis, 
relativo al tiempo requerido por el ``método de barrido'' 
utilizado de manera ubicua en tomografía óptica.
 
\pagestyle{empty}

%\end{document}

