\appendix
\chapter{Apéndice}%
\lhead{\thepage}
\rhead{\textit{Apéndice}}
\vspace{0.01\textheight}
%%%%%%%%%%%%%%%%%%%%%%%%%%%%%%%%%%%%%%%%%%%%%%%%%%%%%%%%%%%%%%%%%%%%%%%%
\section{La aproximación de difusión}
%%%%%%%%%%%%%%%%%%%%%%%%%%%%%%%%%%%%%%%%%%%%%%%%%%%%%%%%%%%%%%%%%%%%%%%%
\label{ap:ecdiff}

El modelo físico mas utilizado para el modelado del transporte de fotones en el 
tejido biológico se basa en la utilización de la aproximación de 
difusión a la ETR~\eqref{eq:RTE}. La razón por la cual este modelo es más utilizado 
que la ecuación de transporte radiativo, es que permite obtener valores 
aproximados para el flujo de fotones, resolviendo un problema matemáticamente 
más simple (la ecuación de difusión), y con un costo computacional 
considerablemente menor que el que exige la resolución de la ETR. 
Si se considera el problema ETR dependiente del tiempo 
en tres dimensiones espaciales, 
este posee 6 variables independientes, donde a las tres variables espaciales $\x=(x,y,z)$, 
deben agregarse las direcciones de propagación en la esfera unitaria $S^2$, las 
cuales para ser determinadas, exigen de dos variables direccionales, con $\hth=\hth(\theta,\varphi)=\cos(\varphi)\sin(\theta)\hat x+\sin(\varphi)\sin(\theta)\hat y+\cos(\theta) \hat z$, y adicionalmente se tiene la variable temporal $t$. En 3D, utilizaremos 
la integración de ángulo sólido, con el elemento diferencial definido según 
$d\theta= d\varphi \sin(\tilde{\theta}) d\tilde{\theta}$, 
donde ahora utilizamos la variable $\tilde{\theta}$ para la integración en el angulo polar, 
para distinguirla de la variable de integración azimutal que usamos en los problemas de simetría 2D, y de la nueva variable de angulo sólido $d\theta$ que utilizamos en esta sección. Bajo la aproximación 
de difusión se reduce la dimensionalidad del problema, que pasa a depender en sólo 4 variables. 
Esto es posible si se asume que, para medios donde domina la dispersión, 
la intensidad específica $u(\x,\hth,t)$ será aproximadamente 
isótropa (\ie, variará suavemente en la variable $\hth$), 
de forma tal que se eliminará la dependencia direccional. Para que la 
ecuación de difusión sea una buena aproximación a la ecuación de 
transporte, se exige que $b(\x)\gg a(\x)$. Cabe destacar que esta aproximación 
es valida en el interior del dominio. Cerca de los bordes, como vimos en la sección~\ref{sec:blayer}, existirán capas límite exponenciales. 
En particular, en las cercanías del borde del dominio, la intensidad específica no será isótropa, 
aún si las condiciones de borde y las fuentes consideradas lo son (ver fig.~\ref{fig:ansol} 
en la sección~\ref{sec:blayer}, y la discusión allí dada). Tampoco será válida esta 
aproximación en la cercanía de fuentes colimadas, que presenten variaciones no lineales en la variable 
direccional $\hth$. Por este motivo, si bien no haremos las consideraciones correspondientes 
en detalle, debe tenerse en cuenta que la aproximación de difusión que discutiremos, 
será válida en el interior del dominio $\Omega$, y lejos de fuentes no isótropas. 

Existen diferentes formas 
de llegar a la ecuación de difusión. Una de ellas se basa en  una expansión perturbativa~\cite{Larsen1974,Larsen1987,Arridge2009}, 
donde se exige que la dispersión sea dominante. En particular, se exige qué $b(\x) \gg a(\x)$, 
y que el camino libre medio de dispersión, definido como $l_b=b(\x)^{-1}$, cumpla $l_b\ll \mathcal{L}$, 
donde $\mathcal{L}$ denota una distancia característica del problema en consideración. 
Este último requerimiento, físicamente implica que para las distancias del problema (donde $\mathcal{L}$ 
podría ser por ejemplo, la dimensión de los lados de los dominios $\Omega$ considerados anteriormente) las partículas realizarán un número muy grande de colisiones, entrando, por lo tanto, en el régimen difusivo. La expansión perturbativa se realiza utilizando las variables 
reescaleadas $a(\x) \rightarrow \varepsilon a(\x)$, $b(\x) \rightarrow \frac{b(\x)}{\varepsilon}$, $s(\x,\hth,t)\rightarrow  \varepsilon s(\x,\hth,t)$ y $t\rightarrow \frac{b(\x)}{\varepsilon}$, con $\varepsilon \ll 1$. Estas variables  expresan físicamente que la dispersión es el fenómeno dominante para las 
escalas espaciales y temporales consideradas. Reemplazando en la ec.~\eqref{eq:RTE} estas variables, junto con la expansión perturbativa $\ut=u_0(\x,\hth,t)+\varepsilon u_1(\x,\hth,t)+\varepsilon^2 u_2(\x,\hth,t)+\ldots$ e igualando los términos del mismo orden en $\varepsilon$ 
hasta el orden cuadrático, se obtiene la ecuación de difusión. Deben realizarse consideraciones especiales con respecto a las condiciones de 
borde que son tratadas en la literatura, y que no detallaremos aquí. Cabe destacar que en la proximidad de los bordes, las partículas no habrán realizado suficientes colisiones para entrar en el régimen difusivo, por lo cual, nuevamente el análisis detallado vale en el interior del dominio espacial $\Omega$. 

Existen otras formas de derivar la ecuación de difusión para los fotones. 
Si se multiplica la versión 3D de la ETR  por la variable angular $\hth$, y se integra 
sobre esa misma variable, se tiene
\begin{equation}
\begin{split}
\begin{aligned}
\frac{1}{c}\frac{\partial}{\partial t} \jc+\int_{S^2} d\theta \nabla & \ut \cdot \hth \hth 
+[a(\x)+b(\x)]\jc=\\
&b(\x)\int_{S^2} \hth d\theta \int_{S^2} \eta(\hth \cdot \hth')  u(\x,\hth',t) d\theta' + s_1(\x,t),
\end{aligned}
\end{split}
\label{eq:RTEjflux}
\end{equation}
donde $s_1(\x,t)$ viene dado por el primer momento de la fuente, y utilizamos 
la versión 3D de la corriente de fotones (ec.~\eqref{eq:photoncurrent})
\begin{equation}
\begin{split}
\begin{aligned}
s_1(\x,t)&=\int_{S^2} \hth s(\x,\hth,t) d\theta,\\
 \jc&=\int_{S^2} \hth \ut d\theta,
\label{eq:RTEsflux1}
\end{aligned}
\end{split}
\end{equation}
donde considerando el caso mas general de tres dimensiones espaciales, integramos en la esfera unitaria $S^2$ donde quedan definidas todas las direcciones posibles de propagación en 3D. 
Utilizando la identidad $\hth= \hth'(\hth\cdot \hth') + \hth' \times(\hth \times \hth')$ reescribimos la integral
\begin{equation}
\begin{split}
\begin{aligned}
\int_{S^2} \hth d\theta \int_{S^2} \eta(\hth \cdot \hth')  u(\x,\hth',t) d\theta' 
&=\int_{S^2} u(\x,\hth',t) d\theta'  \int_{S^2} \hth \eta(\hth \cdot \hth') d\theta\\
&=\int_{S^2} u(\x,\hth',t) d\theta' \int_{S^2} \hth' (\hth \cdot \hth') \eta(\hth \cdot \hth') d\theta \\&\quad+ \int_{S^2} u(\x,\hth',t) d\theta' \int_{S^2}  \hth' \times(\hth \times \hth') \eta(\hth \cdot \hth') d\theta ,
\end{aligned}
\end{split}
\label{eq:RTEjint}
\end{equation}
Debido a que la función de fase sólo depende del ángulo entre 
la dirección incidente y la dirección en la que el fotón es dispersado ($\alpha=\hth \cdot \hth'$), la última integral se anula, de forma qué 
\begin{equation}
\begin{split}
\begin{aligned}
\int_{S^2}  \hth' \times(\hth \times \hth') \eta(\hth \cdot \hth') d\theta =0,
\end{aligned}
\end{split}
\label{eq:RTEjint2}
\end{equation}
Reordenando los términos, y usando qué
el factor de anisotropía viene dado por 
\begin{equation}
g=\int_{S^2} \hth \cdot \hth' \eta(\hth \cdot \hth') d\theta,
\label{eq:gnorm}
\end{equation}
se tiene 
\begin{equation}
\begin{split}
\begin{aligned}
\int_{S^2} \hth d\theta \int_{S^2} \eta(\hth \cdot \hth')  u(\x,\hth',t) d\theta' 
=g \jc
\end{aligned}
\end{split}
\label{eq:RTEjint}
\end{equation}

Llamando al segundo momento de $\ut$
\begin{equation}
\bar{\mathcal{J}}_2(\x,t)=\int_{S^2} \hth \hth \ut d\theta,
\label{eq:RTEjflux2}
\end{equation}
la ecuación~\eqref{eq:RTEjflux} puede reescribirse como
\begin{equation}
\begin{split}
\begin{aligned}
\frac{1}{c}\frac{\partial}{\partial t} \jc+\nabla \cdot \bar{\mathcal{J}}_2(\x,t)+[a(\x)+b'(\x)]\jc=s_1(\x,t),
\end{aligned}
\end{split}
\label{eq:RTEjfluxf}
\end{equation}
donde el coeficiente de dispersión reducido viene dado por $b'(\x)=(1-g)b(\x)$.

Adicionalmente, tenemos de la ec.~\eqref{eq:RTEscflux} para el flujo escalar, la cual derivamos en la 
sección~\ref{sec:ETR} para el caso 2D, que reescribimos aquí considerando un domino $\Omega \in \mathbb{R}^3$
\begin{equation}
\frac{1}{c}\frac{\partial}{\partial t} \phi(\x,t)+\nabla \cdot \vec{ \mathcal{J}}(\x,t)
+a(\x)\phi(\x,t)=s_0(\x,t).
\label{eq:RTEscfluxap}
\end{equation}
Como se mencionó anteriormente, esta ecuación representa la conservación 
de los fotones para cada punto espacial $\x \in \Omega$, independientemente 
de la dirección. Hasta este punto no hemos realizado ninguna aproximación.

El enfoque utilizado con mayor frecuencia en la literatura de tomografía óptica para la derivación de la aproximación de difusión se basa 
en la denominada aproximación $P_N$ de la ETR~\cite{Arridge2009}. Dado que la dispersión tiende a promediar la función de distribución $\ut$ con respecto a la variable $\hth$, volviéndola isótropa, es esperable que en el régimen difusivo la función $u(\x,\hth,t)$ varíe suavemente como función de la variable angular $\hth$. En general, como se vio en la sección~\ref{sec:blayer} esto no será cierto en la proximidad de los bordes. Lejos de los bordes, la variación suave de $\ut$ en $\hth$ para problemas difusivos invita a realizar la expansión en armónicos esféricos
\begin{equation}
\begin{split}
\begin{aligned}
u(\x,\hth,t)&\sim \sum_{l=0}^{N}\sum_{m=-l}^lu_{l,m}(\x,t)Y_{l,m}(\tilde{\theta},\varphi), \\ 
u_{l,m}(\x,t)&=\int_{0}^{2\pi} d\varphi \int_0^{\pi} d\tilde{\theta} \sin (\tilde{\theta})  u(\x,\tilde{\theta},\varphi,t) Y_{l,m}(\tilde{\theta},\varphi) ,
\end{aligned}
\end{split}
\label{eq:PN}
\end{equation}
donde $Y_{l,m}(\theta,\varphi)$ representa al armónico esférico de grado $l$ y orden $m$~\cite{Sansone1991}. La relación~\eqref{eq:PN} valdrá, en general, para $N \rightarrow \infty$. Cuando, debido a la alta dispersión, la función $u(\x,\hth,t)$ 
varíe suavemente en la variable $\hth$, un truncamiento con un valor 
finito de $N$ en la ec.~\eqref{eq:PN} brindará una buena aproximación 
a la solución de la ETR.
Reemplazando esta propuesta en la ecuación ETR lleva a un sistema de ecuaciones acopladas, cuyo primer orden, bajo ciertas aproximaciones, da como resultado la ecuación de difusión. También puede obtenerse la aproximación de difusión considerando las ecuaciones~\eqref{eq:RTEjfluxf} y ~\eqref{eq:RTEscfluxap}. Si bien se espera que, debido a la dispersión, la intensidad específica sea una función isótropa, la dependencia angular no puede ser constante, ya que en tal caso 
no podría haber flujo neto de radiación~\cite[cap. 9, p. 176]{Ishimaru1978}. Por esta razón consideraremos la expansión~\eqref{eq:PN} con $N=1$. 
Es fácil mostrar qué
\begin{equation}
\begin{split}
\begin{aligned}
u_{0,0}(\x,t) Y_{0,0}(\tilde{\theta},\varphi) &=\frac{\phi(\x,t)}{4\pi},\\
\frac{4\pi}{3} \sum_{m=-1}^1 u_{1,m}(\x,t)Y_{1,m} (\tilde{\theta},\varphi) & = \frac{3}{4\pi} \jc \cdot \hth,
\end{aligned}
\end{split}
\label{eq:PN2}
\end{equation}
De donde se tiene de~\eqref{eq:PN} a primer orden
\begin{equation}
\begin{split}
\begin{aligned}
u(\x,\hth,t)&\sim \frac{\phi(\x,t)}{4\pi}+\frac{3}{4\pi} \jc \cdot \hth.
\end{aligned}
\end{split}
\label{eq:PN3}
\end{equation}
Reemplazando~\eqref{eq:PN3} en~\eqref{eq:RTEjflux2} tenemos
\begin{equation}
\bar{\mathcal{J}}_2(\x,t)=\int_{S^2} \hth \hth \left(  \frac{\phi(\x,t)}{4\pi}+\frac{3}{4\pi} \jc \cdot \hth \right) d\theta,
\label{eq:RTEjflux22}
\end{equation}
Dado que la corriente $\jc$ no depende de la variable $\hth$, las integrales que 
constituyen los elementos del tensor de segundo orden $\bar{\mathcal{J}}_2(\x,t)$ están formadas 
por productos de potencias de funciones trigonométricas que pueden ser evaluadas fácilmente. 
Es fácil verificar que la integral en el segundo término del lado derecho de la ecuación~\eqref{eq:RTEjflux22} se anula, y qué~\cite[cap. 17, p. 544]{Modest2013}
\begin{equation}
\int_{S^2} \hth \hth d\theta=\frac{4\pi}{3} \id,
\label{eq:RTEjflux22}
\end{equation}
de donde se tiene 
\begin{equation}
\nabla \cdot \bar{\mathcal{J}}_2(\x,t)=\frac{1}{3}\nabla \phi(\x,t).
\label{eq:RTEjflux23}
\end{equation}
Asumiendo que la fuente $s(\x,\hth,t)$ es isótropa, se tendrá $s_1(\x,t)\simeq 0$. Reemplazando estos resultados en la ecuación~\eqref{eq:RTEjfluxf}
\begin{equation}
\begin{split}
\begin{aligned}
\frac{1}{c}\frac{\partial}{\partial t} \jc+\frac{1}{3}\nabla  \phi(\x,t) +[a(\x)+b'(\x)]\jc=0,
\end{aligned}
\end{split}
\label{eq:RTEjfluxf3}
\end{equation}
Además, considerando que el tiempo que le toma a los fotones en recorrer 
el camino libre medio ocurre en escalas temporales mucho 
mas pequeñas que las consideradas para las variaciones temporales de $\jc$, se 
tiene qué $\frac{1}{c}\frac{\partial}{\partial t} \jc \simeq 0$, de donde de 
la ec.~\eqref{eq:RTEjfluxf3} se llega a la Ley de Fick
\begin{equation}
\begin{split}
\begin{aligned}
\jc=-\frac{1}{3[a(\x)+b'(\x)]} \nabla  \phi(\x,t) .
\end{aligned}
\end{split}
\label{eq:Fick}
\end{equation}
donde el signo negativo expresa que el flujo de fotones se da desde regiones 
de mayor densidad (o ``concentración'') hacia regiones de menor densidad de fotones. 
Definiendo el coeficiente de difusión $D(\x)=\frac{1}{3[a(\x)+b'(\x)]}$, y reemplazando 
este último resultado en la ecuación~\eqref{eq:RTEscfluxap}, se 
llega finalmente a la ecuación de difusión para la densidad de fotones
\begin{equation}
\frac{1}{c}\frac{\partial}{\partial t} \phi(\x,t)-\nabla \cdot \left( D(\x) \nabla  \phi(\x,t) \right)
+a(\x)\phi(\x,t)=s_0(\x,t).
\label{eq:difusion}
\end{equation}
Las condiciones de borde que deben ser impuestas en la ecuación~\eqref{eq:difusion} 
merecen un capitulo a parte, y escapan al alcance de esta tesis. Como se mencionó anteriormente, la aproximación de difusión es válida en lejanías 
de los bordes. Se han desarrollado diferentes teorías para el tratamiento de los bordes, 
que incluyen la utilización de bordes extrapolados~\cite{Arridge2009,Haskell1994,Arridge1999,Xu2002}. En general, en la bibliografía 
no se discute el tratamiento adecuado de las capas límite discutidas en la sección~\ref{sec:blayer}. Dado que en tomografía óptica las fuentes y los detectores 
se ubican en el contorno del dominio a analizar, es esperable que una aproximación 
que no es apropiada en estas regiones tenga un impacto negativo significativo en las reconstrucciones obtenidas en problema inverso basado en la ecuación~\eqref{eq:difusion}. 


%%%%%%%%%%%%%%%%%%%%%%%%%%%%%%%%%%%%%%%%%%%%%%%%%%%%%%%%%%%%%%%%%%%%%%%%
\section{El algoritmo FC(Gram)}
%%%%%%%%%%%%%%%%%%%%%%%%%%%%%%%%%%%%%%%%%%%%%%%%%%%%%%%%%%%%%%%%%%%%%%%%
\label{ap:ecdiff}

